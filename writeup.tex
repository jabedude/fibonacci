\documentclass{article}
\usepackage{amsmath}
\usepackage{listings}
\usepackage[margin=1in]{geometry}

\usepackage{color}
 
\definecolor{codegreen}{rgb}{0,0.6,0}
\definecolor{codegray}{rgb}{0.5,0.5,0.5}
\definecolor{codepurple}{rgb}{0.58,0,0.82}
\definecolor{backcolour}{rgb}{0.95,0.95,0.92}
 
\lstdefinestyle{mystyle}{
    backgroundcolor=\color{backcolour},   
    commentstyle=\color{codegreen},
    keywordstyle=\color{magenta},
    numberstyle=\tiny\color{codegray},
    stringstyle=\color{codepurple},
    basicstyle=\footnotesize,
    breakatwhitespace=false,         
    breaklines=true,                 
    captionpos=b,                    
    keepspaces=true,                 
    numbers=left,                    
    numbersep=5pt,                  
    showspaces=false,                
    showstringspaces=false,
    showtabs=false,                  
    tabsize=2
}
 
\lstset{style=mystyle}

\begin{document}


\centerline{\sc \large x86 Assembly Writeup}
\vspace{.5pc}

\begin{flushleft}
\textbf{Name:} Joshua Abraham
\vspace{.5pc}

\textbf{Date:} 9 November 2017
\vspace{.5pc}

\textbf{Current Module:} x86 Assembly
\vspace{.5pc}

\textbf{Project Name:} "Fibonacci"
\vspace{.5pc}

\textbf{Project Goals:}
\vspace{.5pc}
\end{flushleft}

The fibonacci project aims to create a program written in x86 assembly that calculates the 
n\textsuperscript{th} Fibonacci number, where 0 $\leq$ n $\leq$	500.
\vspace{.5pc}

\begin{flushleft}
\textbf{Considerations:}
\vspace{.5pc}
\end{flushleft}

\begin{itemize}
	\item[$\bullet$] The program must take one command line parameter, n.
	\item[$\bullet$] The program should produce an error message if no 
	parameter is passed.
	\item[$\bullet$] The program must be able to calculate Fibonacci numbers 
	from 0 $\leq$ n $\leq$ 500.
\end{itemize}
\vspace{.5pc}

\begin{flushleft}
\textbf{Initial Design:}
\vspace{.5pc}
\end{flushleft}

The project is composed of the following files:
\begin{itemize}
	\item [$\cdot$] \textit{Makefile}: The main makefile for the project.
	\item [$\cdot$] \textit{fibonacci.s}: The source code for fibonacci.
	\item [$\cdot$] \textit{test.sh}: The test runner for the project.
\end{itemize}
\vspace{2mm}

\begin{flushleft}
\textbf{Data Flow:}
\vspace{.5pc}
\end{flushleft}

My fibonacci program begins execution by performing command-line argument 
checks and string to integer conversion.  The first check is on the number of 
arguments.  The second check occurs during the string conversion routine.  If 
the user input begins with a '-', the program exits and prints a usage 
statement.  Next, it finishes converting the input into an integer. The 
program then enters the Fibonacci algorithm shown in Figure 1:
\begin{equation}
    f(n) = \begin{cases}
               0               & n = 0\\
               1               & n = 1\\
               f(n-1) + f(n-2) & \text{otherwise}
           \end{cases}
\end{equation}
If the input is zero or one, the program handles these "Edge cases" in
Edge\_fib and exits cleanly.  Otherwise the program continues into a loop 
where it uses three six-QWORD chunk numbers to calculate a desired Fibonacci
number.

\begin{lstlisting}[language=C, caption=Fibonacci in C]
long num, one = 0, two = 1, fib_num;
for (int i = 1; i < num; i++) {
            fib_num = one + two;
            one = two;
            two = fib_num;
}
\end{lstlisting}
Code listing 1. shows the Fibonacci algorithm that the program uses to calculate 
$f(n)$ where 2 $\leq$ n $\leq$ 500.  The program uses three six-QWORD chunk 
variables to calculate each successive Fibonacci number.  Using the 
traditional rcx register to count until it reaches the user input, the program
follows the structure of the C code in listing 1.  Once the number is 
calculated, the program prints the number using a call to the C function 
\textit{printf}, then returns 0 and exits. 
\vspace{.5pc}

\begin{flushleft}
\textbf{Communications Protocol:}
\vspace{.5pc}
\end{flushleft}
No communication occurs besides from user to program (command-line) and 
program to user (STDOUT).
\vspace{.5pc}

\begin{flushleft}
\textbf{Potential Pitfalls:}
\vspace{.5pc}
\end{flushleft}

\begin{itemize}
	\item[$\bullet$] Calculating extremely large numbers (well over 100's of 
	septendecillions)
	\item[$\bullet$] Keeping assembly code well organized
	\item[$\bullet$] Over reliance on C functions.
	\item[$\bullet$] Flow control of program without high-level constructs 
	such as while/for loops.
\end{itemize}
\vspace{.5pc}

\begin{flushleft}
\textbf{Test Plan:}
\vspace{.5pc}
\end{flushleft}

\textit{User Tests:}
\begin{itemize}
	\item[$\cdot$] Performed test specified in the project supplement manual.
	\item[$\cdot$] Gave malformed input (strings, more than one argument, 
	negative numbers, numbers larger than 500).
	\item[$\cdot$] Gave no input.
\end{itemize}

\textit{Test Cases:}
\begin{itemize}
	\item[$\cdot$] Used script to automate testing.
	\end{itemize}
\vspace{.5pc}

\begin{flushleft}
\textbf{Conclusion:}
\vspace{.5pc}
\end{flushleft}
The Fibonacci project has been a challenging experience.  Lack of high-level
loops, conditionals, etc. made implementing a program that calculates the 
Fibonacci sequence hard. Having to calculate up to the 500\textsuperscript{th}
number (an absurdly large value) made it even harder.  However, this project 
was a resounding success.  All requirements were accomplished, with minimal
C function calls.  The program is fast, easy to read, and well documented.  I
learned a great deal during development; one of the most practical lessons was
being able to translate high-level C into assembly almost line by line.  If I 
had more time, I would spend it on removing my call to \textit{printf} 
entirely and use 100\% assembly.

\end{document}